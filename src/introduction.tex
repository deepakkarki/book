
\section{Introduction}
\label{chap:introduction}

Rewrite whole introduction, motivations have changed too much to tweak
the old one.

\subsection{Book Outline}

The remainder of the book is as follows.  \chapref{concept} defines the
concept of the anykernel and rump kernels.  \chapref{implementation}
discusses the implementation and provides microbenchmarks as supporting
evidence for implementation decisions.  Essentially the two previous
chapters are a two-pass flight over the core subject.  The intent is to
give the reader a soft landing by first introducing the new concept in
abstract terms, and then doing the same in terms of the implementation.
That way, we can include discussion of worthwhile implementation details
without confusing the high-level picture.  If something is not clear from
either chapter alone, the recommendation is to study the relevant text from
the other one.  \chapref{ecosystem} gives an overview of what we have
built on top of rump kernels.  A brief history of the project is given
in \chapref{history} and \chapref{conclusions} provides concluding remarks.


\subsection{Further Material}

\subsubsection{Source Code}
\label{sect:src}

The basis of rump kernels is formed by the NetBSD operating system.
For example, the majority of the drivers and the general infrastructure
is directly used from the NetBSD source repository.
The NetBSD files histories are available for study from repository
provided by the NetBSD project, \eg via the web interface at
\texttt{cvsweb.NetBSD.org}.

Additionally, there is infrastructure to support building rump kernels
for various platforms, hosted at \texttt{http://repo.rumpkernel.org/}.

The easiest way to fetch the NetBSD source code is to run the following
commands:

\begin{verbatim}
git clone http://repo.rumpkernel.org/src-netbsd
cd src-netbsd
git checkout all-src
\end{verbatim}

Unlike in the original dissertation, we do not define exact revisions
for the source code we describe in this book.  However, we attempt to
keep all discussion in such a state that it is fully up-to-date whenever
a new version of the book is released.

\subsubsection*{Code examples}

This book includes code examples from the NetBSD source
tree.  All such examples are copyright of their respective owners
and are not public domain.  If pertinent, please check the full
source for further information about the licensing and copyright
of each such example.

\subsubsection{Manual Pages}

Various manual pages are cited in the document.  They are available
as part of the NetBSD distribution, or via the web interface
at \texttt{http://man.NetBSD.org/}.
