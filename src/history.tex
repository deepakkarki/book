
\section{Short History}
\label{chap:history}

This chapter gives a short overview of the history of rump kernels and the
events that lead to the current situation.  The intention is to provide
reasonable amounts of insight into why things evolved like they did.
This chapter is written in chronological order on a high level, but we
attempt to discuss one subject matter in one go, so there is no timeline
in the strict sense.

We will use the late 2006 to early 2007 period as the starting point
for the story.  Back then, work was done on the \textit{puffs} userspace
file systems framework in NetBSD.  This work led to key observation for
rump kernels: it is very easy to implement a superficially working file
server in userspace because a crash does not mean the end of everything.
However, if the file server is supposed to be both performant and robust,
things get very complicated even in userspace.  In effect, the difference
that userspace offers is the immeasurably nicer development environment.

The next logical question to raise was if there is any fundamental reason
that kernel file systems cannot be developed in userspace without having
to perform back-and-forth porting of the drivers between userspace
interfaces and kernel interfaces.  Given that you are reading this
document, the answer is: no.


\subsection{First Steps}

In summer 2007, running NetBSD's kernel FFS driver as a puffs userspace
server was made possible.  The guideline for the work was that the FFS
driver must remain unmodified, and various shims be built for linking and
running the driver.  Making the unmodified FFS driver work in userspace
took approximately two weeks.  A good part of that time was spent
fixing bugs in the shim that translated file offsets to backing device
block numbers (\textit{getpages}, \textit{putpages} \& \textit{bmap}).
Translating offsets sounds simple enough, but when you have to take
into account the sizes of device blocks, file system blocks and memory
pages, along with potentially extending the file, things start getting
complicated.

Originally, the working name for rump kernels was \textit{sakern}, which
was officially supposed to mean ``standalone kernel''.  Unofficially, it
was supposed to be a reference to the Swedish language and mean
``things'', denoting the slightly messy nature of the shims.  The name
was changed to \textit{RUMP} before import.  Later, that ``backronym''
would be dropped in favor of ``rump kernels''.  For consistency, we will
simply use ``rump kernels'' throughout this section, instead of altering
the nomenclature based on which timeframe we are talking about.

In August 2007, rump kernels were first introduced into the NetBSD
source tree.  At that time, there was support for a number of file
systems which could run as puffs servers.  Support for running the file
drivers as standalone programs was yet not available, since there was no
support for the top layers of the kernel.  For example, the \textit{namei}
part of the \textit{namei+lookup} ping pong game was handled by puffs
in the host kernel, and rump kernels only supported \textit{lookup}
performed by individual file system drivers.

Also around the same time, the first steps were taken to run NetBSD
kernel code on non-NetBSD hosts.  Some assumptions about the host being
NetBSD were addressed and fixed, and a mental note was made about what was
required for supporting non-NetBSD platforms.  At that time, building for
non-NetBSD required huge amount of manual work, and it would not be until
late 2012 when \textit{buildrump.sh} was introduced that building any
given vintage of rump kernels for non-NetBSD platforms would be made easy.


\subsection{Towards robust and maintainable}

Hacking up something that runs is not difficult.  The real challenge is
in making things maintainable in a system under constant development.
While the shims made it possible to run kernel drivers from one vintage of
NetBSD, the shims would eventually go out-of-date and cause compile-time
and runtime failures.  Since NetBSD kernel code will not go out-of-date
with itself, shims were slowly replaced by using NetBSD kernel code
directly.  Figuring out which interfaces need to implemented as shims,
and which interfaces can be used directly from NetBSD sources ended up
being a multi-year effort.  The general principles on what to reuse and
what to rewrite have been more or less clear since 2011.

Another area that needed improvements was multithreading support.
Initially, rump kernels did not support locking, synchronization, or
anything that interacted with the scheduler.  This was fine for file
system drivers, because most of them use only a single thread, and
simply ignoring most of the aspects of locking was enough.  The problem
with that approach was of course that it was not good enough to support
any given kernel driver, and neither could it be used to exercise the
multithreading robustness of file system drivers.

Most of the support for multithreading was added later in 2007.  Support
for kernel threads and locking and was relatively straightforward to add.
The hard part of multithreading support was figuring out how the inverse
scheduling model should be handled --- normally operating systems pick a
thread to schedule on a core, but rump kernels schedule a virtual core for
a thread.  The original design of the scheduler was done on an airplane
napkin in 2009, but coming up with a deadlock free implementation was
anything but trivial.  It took over a year to successfully implement
the design.


\subsection{Syscall support}

As mentioned, in the beginning rump kernels supported only file servers,
and those accessed the kernel drivers directly at the in-kernel virtual
file system layer.  Hence, there was no need to support the POSIX-style
system call interfaces.

However, as application possibilities such as \textit{fs-utils}
were being imagined, the necessity for a syscall-like interface grew.
One early attempt at a syscall-like facility was \textit{ukfs} (or
``userspace kernel file system'').  It implemented enough of the top half
of the kernel for syscall-like pathname-based access to file systems to
be possible.

While ukfs worked for file systems, it had issues:
\begin{enumerate}
\item	The implementation of the top half, especially \textit{namei},
	is complex, so getting it right was non-trivial.
\item	The interface was not quite syscall-like, which necessitated
	writing applications specifically for ukfs instead of being able
	to use existing applications against rump kernels in a library-like
	fashion.
\item	ukfs was specific to file systems and did not address
	other subsystems such as the networking stacks.
\end{enumerate}

To remedy this, the top levels of the kernel, including a subset of
the system call handlers and namei, were added to rump kernels in
the beginning of 2008.  One mistake was originally made with the
introduction of the rump kernel syscall interface.  To avoid the
implicitly thread-local quality of the \texttt{errno} variable, each
system call interface in the rump kernel syscall API took an extra
``\verb+int *error+'' parameter for returning the error.  This approach
made it difficult to use existing applications against the rump
kernel syscall API, since the application side required modification.
It took almost a year for work to progress far enough for the extra
parameter to become a problem.  The extra error parameter was removed
in January 2009 now making the rump kernel syscall API and ABI equal to
the one provided by the NetBSD libc.

As for ukfs, the homegrown pieces were removed, and the interface was
implemented on top of rump kernel syscalls.  For some years, ukfs was
intended to become a portable interface for accessing file systems,
\ie one which does not suffer from type incongruence between the rump
kernel and the host (think of the canonical ``\verb+struct stat+''
example).  However, now in 2014 ukfs is considered all but an obsolete
early attempt on the road to figuring out how to best do things.


\subsection{Beyond file systems}

From the start, the idea was to support multiple kernel subsystems with
rump kernels.  This is evident also from the directory structure of the
initial 2007 import: file systems were in \texttt{sys/rump/fs}.

It took over a year from the initial import in 2007 for support for
another subsystem to appear.  The natural choice was networking, since
it was required by NFS, and support was added late 2008.  As part of
the initial networking support, a virtual network interface to access
the network via the host's \textit{Ethernet tap} device was added.
The idea was to provide network access, not to necessarily be fast.
Support for the fast userspace packet I/O frameworks DPDK and
netmap emerged in 2013 --- not that those frameworks even existed back
in 2008.

Beyond file systems and networking, the remaining major source of
drivers is for devices.  These were the last major set of drivers
for which rump kernel support was added.  The first steps were taken
in mid-2009 with the support of the kernel autoconfiguration framework
in rump kernels.  Later in 2009, the first experiments with driving
actual devices was taken in the form USB devices.  Even so, the USB
implementation did not touch raw devices, and depended on the USB host
controller driver being supported by the host.  Support for real device
drivers appeared first in 2013 with PCI device support under Xen.


\subsection{Symbol isolation}

If kernel code is compiled and linked into a userspace program, there is a
danger of symbol collisions.  For example, both the kernel and userspace
have a function called \verb+malloc()+.  However, the kernel malloc
of NetBSD has a different signature, and takes \eg a flag parameter
which specifies whether or not the returned memory should be zeroed.
In early versions of rump kernels, kernel and userspace symbols
were haphazardly linked together.  For example, if the kernel malloc
is serviced by the libc implementation, the flags will be ignored and
potentially non-zeroed memory will be returned.  Amusingly enough,
it took a few months before a bug was triggered by malloc not zeroing
memory even if it was expected to.

Early attempts at avoiding symbol collisions were done in an adhoc
fashion.  For example, malloc was dealt with by using the \verb+--wrap+
flag in the linker, and for example the kernel \verb+valloc()+ routine
was renamed the \verb+vnalloc()+ to avoid a collision with userspace.
The exact set of conflicting symbols, however, depend on the platform,
so the early attempts were never considered anything more than bandaid.

The real solution came in early 2009 with mass symbol renaming:
at build-time, all symbols in a rump kernel are mass renamed to start
with the prefix ``rump''.  While doing so solved the collision problem,
the renaming also had a much more profound implication.  After symbol
renaming was instated, the rump kernel was now a closed namespace; a rump
kernel could only access routines within itself or in the hypercall layer.
We now know for certain that a rump kernel will not block without making
a hypercall.


\subsection{Local clients}

One of the problems of the scheduling model used by rump kernels is
where and how to decide which process and thread context the currently
executing host thread possesses in the rump kernel.  Originally, and
heavily influenced by the original file server mode of operation, it was
completely up to the client to specify the PID and thread ID.  Having the
client select the PID allowed for the file server to operate with the
same process context as the original operation in the host kernel.
Having the same PID meant that errors produced by the fs driver in the
rump kernel, \eg ``file system full'', reported the correct host PID
that had caused the malfunction.

The downside of the ``client decides'' approach was that the client
had to decide even if it did not want to, \ie there was no simple way
for the client to say ``I want a new process for a new set of file
descriptors, don't care about the PID''.  The end result of all of this
was that emulating things like fork and exit for remote clients would
have been close to impossible.

Figuring out a sensible interface for deciding the current context
took many years of development and 3 or 4 rewrites of the interface.
In 2010, the \textit{lwproc} set of interfaces was introduced, along with
the concepts of implicit and bound threads, marking a bit milestone in
defining the unique characteristics of rump kernels.


\subsection{Remote clients}

The first experiments with remote clients were done in early 2009.
However, it took over 1.5 years for a revised implementation to appear.
Unlike in the case of the scheduler, the delay was not due to the code
being complicated to figure out.  On the contrary, it took so long
because writing RPC code was boring.

While remote clients showed that it is possible to use Unix as a
distributed OS without requiring a complete Plan~9-like implementation
from scratch, that was not the main goal of remote clients.  Support for
remote clients made it possible to configure a rump kernel in a more
natural way, where a configuration program is run with some parameters,
the program performs some syscalls on the kernel, after which the program
exits.  Before remote clients, any configuration code was run in a local
client, usually written manually.  This was not a huge problem for
file systems, since the configuration is usually just a matter of
calling \verb+mount()+, but for configuring the network stack, not being
able to use \texttt{ifconfig}, \texttt{route} and friends was becoming
an issue.


\subsection{Shifting focus towards driver reuse}

In early 2011, rump kernels were declared complete; the basic
infrastructure required by portable kernel drivers had been figured after
about four years more of more or less fulltime work.  The declaration
also marked the end of most of the work taking place in the NetBSD source
tree, since the main architecture of rump kernels had been defined.
Given that the goal is to stick to unmodified NetBSD source code,
a decent amount of work still goes on directly in the NetBSD tree,
but other development loci are emerging.

The first major landmark was the introduction of the \textit{buildrump.sh}
script in late 2012.  The importance of buildrump.sh was in that for the
time it was simple to build a given NetBSD source vintage for a non-NetBSD
platform.  Buildrump.sh was also the first piece of code to be committed
onto GitHub to what would eventually become the rumpkernel organization.

The shift toward driver reuse also brought a completely new focus
for development.  Earlier, when debugging was the main motivation,
the main focus was making rump kernels convenient.  A good example of
this attitude are the implicit threads provided by the rump kernel
scheduler: if you do not know about bound threads or do not care to
manage them, the default implicit threads will always work correctly,
though they do not perform well.  With driver reuse, and especially with
networking, it started being important to be as performant as possible.
Improving performance is still on-going work, but steps such as improving
the performance of \texttt{curlwp} have been done in 2014.  In fact,
the old and new goals conflict slightly, because with implicit threads
things will work just fine, but unless you know about bound threads,
you will be losing a good deal of potential performance.


\subsection{Going multiplatform}

Up until late 2012, rump kernels were only able to run in userspace
processes.  Support was mostly limited to NetBSD and Linux at the time,
but it's a small hop from one userspace to another (though the amount
of details involved in those small hops is rather dumbfounding).
The challenge was to evaluate if rump kernels could run on top of
literally anything.

The first non-userspace platform ended up being a web browser, with
kernel drivers being compiled to javascript instead of the usual assembly.
That experiment was both a success and a failure.  It was a success
because rump kernels ran smoothly in a web browser when compiled with a
C$\rightarrow$javascript compiler.  It was a failure because the compiler
provided emulation for the POSIX interfaces, and the existing rump kernel
hypercalls could be used.  In essence, a rump kernel running as javascript
in a web browser still worked more or less like when running in userspace.

The second non-userspace experiment was undertaken in early 2013.
It was a bit more successful in choosing a decidedly non-POSIX platform:
the Linux kernel.  There were no big surprises in getting rump kernels
running in the Linux kernel, though the experiment motivated minor changes
to the rumpuser hypercall interface, along with documenting the entire
interface in the rumpuser manual page.

Even though the Linux kernel is not a POSIX environment, it still
resembles userspace a fair deal: there are scheduled threads,
synchronization primitives, etc.  To further test the portability
of rump kernels, a hypercall implementation for the Xen hypervisor
was written in summer 2013.  Since the Xen hypervisor is a barebones
platform without the above mentioned userspace-like synchronization and
multiprocessing features, it would test rump kernel portability even
more than the Linux kernel did.  Granted, the hypercalls were written
on top of the Xen MiniOS instead of directly on top of the hypervisor,
but MiniOS is there only to provide bootstrapping support, access
to paravirtualized devices and MD code for stack switching.  All in all,
the platform is more or less like running rump kernels directly on top
of bare metal.

The Xen port was also unique in the sense that it was the first time
that running a full application stack on top of a rump kernel, including
libc and other userspace libraries, was done.  Earlier, clients used rump
kernels only as libraries, but especially standard applications depended
also on the functionality provided by the host system.  With additional
work put into the Xen port, it became possible to run off-the-shelf
programs fully on top of rump kernels.  The Xen port was also the first
time that a rump kernel was fully in control of the symbol namespace,
instead of integrating into an existing one.

Last but definitely not least, the first third party rump kernel hypercall
implementation was done by Genode Labs for their Genode OS Framework.
Support was first released as part of Genode 14.02, which in a highly
mnemonic fashion was released in February 2014.


\subsection{Rumprun, the breakthrough of portable clients}

Ever since August 2007 when rump kernels were run on Linux for the first
time, the problem of how to access the rump kernel on non-NetBSD platforms
was known.  Before the focus shift in 2012, it was never really a big
issue: things worked on NetBSD and the occasional test on a non-NetBSD
platform limped along as best it could.  Some truly necessary things like
configuring interface addresses were supported with adhoc solutions,
but a generalized solution allowing to use the full set of NetBSD
configuration utilities was missing.

The essence of the problem was that non-NetBSD platforms didn't provide
the types required for configuring the NetBSD kernel (\eg ioctls used
by ifconfig).  Neither was using those types from the NetBSD headers
possible, since trying to include a NetBSD header while compiling on
a non-NetBSD platform just resulted in conflict hell (duplicate header
names, duplicate types, ...).

The innovation that came in late 2013 was to build on top of the work
done with Xen, and run a full NetBSD software stack upon a rump kernel
in userspace.  Since the clients were now essentially compiled as NetBSD
utilities, all of the necessary types were available.  The contribution
that rumprun made was figuring out how to isolate the NetBSD software
stack into a userspace process that so the NetBSD stack would not conflict
with the host.

With rumprun, many things that used to be a seemingly impossible
amount of work became trivial.  One good example is using the NetBSD
\verb+wpa_supplicant+ to configure a wireless card controlled by a rump
kernel on Linux.  The NetBSD version of the utility, along with the
libraries it depends on (\eg OpenSSL) can simply be configured and ran
without breaking a swear.  After a fashion, rumprun can be considered
the userspace counterpart of rump kernels, allowing highly specific code
to run in any environment without porting efforts.


\subsection{Summary}

The initial implementation of rump kernels built around file system
drivers was minimalistic.  Throughout the years the interfaces and
implementation were modified to achieve both better clientside usability
and maintainability.  Introducing features and dependencies was avoided
where feasible by critically examining each need.


\subsection{TODO}

\begin{itemize}
\item	hijack
\end{itemize}
