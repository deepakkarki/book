\documentclass[openright]{tkkdiss}

\newcommand{\code}{\begin{figure}
{\tt \scriptsize
\begin{verbatim}}}

\newenvironment{dedication}
    {\begin{quotation}\begin{center}\begin{em}}
    {\par\end{em}\end{center}\end{quotation}}

\usepackage{nomencl}
\usepackage{mynomencl}
\usepackage{graphicx}
\usepackage{subfigure}
\usepackage{enumitem}
\usepackage{epsfig}
\usepackage{alltt}
\usepackage{afterpage}
\usepackage{fix-cm}
\usepackage{xcolor}
\usepackage{pdfpages}

% LAST PACKAGE
\usepackage[pdfusetitle]{hyperref}


\renewcommand{\ttdefault}{txtt}

\newcommand{\manpageref}[2]{manual page \textit{#2}}
\newcommand{\chapref}[1]{\hyperref[chap:#1]{Chapter~\ref{chap:#1}}}
\newcommand{\eg}{e.g.\ }
\newcommand{\ie}{i.e.\ }

\hyphenation{namespace}
\hyphenation{anykernel}

\makenomenclature

\begin{document}

% cover page graphics
% compensate tkkdiss.cls hvoffset with equal but opposite hvoffset
\includepdf[offset=1in -1in]{cover5b.pdf}

\cleardoublepage

Copyright (c) 2012, 2014 Antti Kantee\\
All Rights Reserved

Credits for the original thesis: \\
Author: Antti Kantee \\
Supervisor: Prof. Heikki Saikkonen \\
Preliminary examiners: Dr. Marshall Kirk McKusick, Prof. Renzo Davoli \\
Opponent: Dr. Peter Tr\"{o}ger \\

{\large PRELIMINARY DRAFT.  DO NOT DISTRIBUTE.}

\begin{preface}

\begin{verse}
\textit{I'll tip my hat to the new constitution\\
Take a bow for the new revolution\\
Smile and grin at the change all around\\
Pick up my guitar and play\\
Just like yesterday\\
Then I'll get on my knees and pray\\
We don't get fooled again\\
-- The Who}
\end{verse}

This document is intended as an up-to-date description on the fundamental
concepts related to the anykernel and rump kernels.  It is based on the
dissertation written in 2011 and early 2012: \textit{Flexible
Operating System Internals: The Design and Implementation of the Anykernel
and Rump Kernels}.

The major change with rump kernels since the first edition is a shift
in focus and motivation.  In work leading up to the first edition, rump
kernels were about running kernel components in userspace.  Since then,
it has become possible to run complete POSIX-environment application
stacks on top of rump kernels in most environments, including userspace,
Xen, Genode and even bare metal.

The second edition will also limit itself to being an architecture
description, and discussion on use cases of have been removed.  The reader
is invited to look at \textit{http://repo.rumpkernel.org/} and
\textit{http://wiki.rumpkernel.org/}.  Furthermore, some of the
measurements mandated by the original text's academic nature are gone.

However, the second edition is currently not yet finished.  It is
recommended that you read the original dissertation which is available
via \textit{http://book.rumpkernel.org/}.

\end{preface}


\tableofcontents


\begin{listofabbreviations}
\input{abbr.tex}
\printnomenclature[5cm]

\end{listofabbreviations}


\listoffigures
\listoftables


\input{inputs.tex}


\bibliographystyle{tkkdiss}
\bibliography{thebibs}


\end{document}
