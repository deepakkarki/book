
\section{Conclusions}
\label{chap:conclusions}

A codebase's real value lies not in the fact that it exists, but that
it has been proven and hardened ``out there''.  The purpose of this work
is harnessing the power and stability of existing in-kernel drivers.

We claimed that a properly architected and maintained monolithic kernel
provides a flexible codebase.  We defined an \textit{anykernel} to be an
organization of kernel code which allows the kernel's \textit{unmodified}
drivers to be run in various configurations such as libraries, servers,
small standalone operating systems, and also in the original monolithic
kernel.  We showed by means of a production quality implementation that
the NetBSD monolithic kernel could be turned into an anykernel with
relatively simple modifications.

The development effort for turning a monolithic kernel into an anykernel
is insignificant next to the total effort a real world capable monolithic
kernel has received.  The anykernel architecture by itself is not a
solution to the problems; instead, it is merely a concept which enables
solving the problems while still retaining the ability to run the kernel
in the original monolithic configuration.

An anykernel can be instantiated into units which include the minimum
support functionality for running kernel driver components.  These units
are called \textit{rump kernels} since they provide only a part of
the original kernel's features.  Features not provided by rump kernels
include for example a thread scheduler, virtual memory and the capability
to execute processes.  There omissions make rump kernels straightforward
to integrate with any platform that has approximately one megabyte or
more of RAM and ROM.

We experimented with a number of platforms for rump kernels, from POSIX
userspace to bare metal and hypervisors and web browsers with rump
kernels compiled to javascript.  Accessing platform resources is done
through the \textit{rumpuser} hypercall interface.

In their maximal configuration, rump kernels support an almost full
POSIX system call interface, with the only major missing interfaces
being those depending on functionality not included in rump kernels,
e.g. \texttt{fork()}.  That said, if rump kernels are run on a platform
which does support functionality like \texttt{fork()}, rump kernels can
be easily be integrated with the functionality provided by platforms,
and calls such as \texttt{fork()} can be emulated.

We also showed that one can optionally run a libc and regular POSIX
applications on top of rump kernels.  The transitive result is the
following: POSIX applications can be run on any platform that can support
rump kernels with the effort of implementing the rump kernel hypercalls.
